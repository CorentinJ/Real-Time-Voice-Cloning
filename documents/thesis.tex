\documentclass[a4paper, oneside]{article}
\usepackage{array}
\usepackage{shortvrb}
\usepackage{listings}
\usepackage{fullpage}
\usepackage{enumerate}
\usepackage{graphicx}
\usepackage{subfigure}
\usepackage{url}
\usepackage{indentfirst}
\usepackage{eurosym}
\usepackage{listings}
\usepackage{color}
\usepackage{fancybox}
\usepackage{ulem}
\usepackage{wrapfig}
\usepackage{systeme}
\usepackage{tabularx}
\usepackage{subfig}
\usepackage[dvipsnames]{xcolor}



\begin{document}
\begin{titlepage}
	\newcommand{\HRule}{\rule{\linewidth}{0.5mm}}
	\center
	\textsc{\LARGE Université de Liège}\\[1cm]
	\textsc{\Large Faculté des Sciences Appliquées}\\[2cm]
		
	\HRule \\[0.5cm]
	{ \huge \bfseries Automatic Voice Cloning Across Languages}\\[0.2cm]
	\HRule \\[3cm]

	\begin{minipage}{0.4\textwidth}
		\begin{flushleft} \Large
			\emph{Author:}\\
			Corentin \textsc{Jemine}
		\end{flushleft}
	\end{minipage}
	~
	\begin{minipage}{0.4\textwidth}
		\begin{flushright} \Large
			\emph{Supervisor:} \\
			Prof. Gilles \textsc{Louppe}
		\end{flushright}
	\end{minipage}\\[4cm]
	
	{\LARGE Academic year 2018 - 2019}\\[2.5cm]
	
	\includegraphics{images/uliege_logo.jpg}\\[1.25cm]
	
	\textit{Graduation studies conducted for obtaining the Master's degree \\in Data Science by Corentin Jemine}
	
	\vfill
\end{titlepage}

\setcounter{page}{2}

\section{Abstract}
\color{red}
To do when I'll have a good overview of the project. Try to answer:
\begin{itemize}
	\item What is the goal of the application? What are its requirements, what is the setting, what kind of data are we going to use it on?
	\item What is zero-shot voice cloning? How does it fit in here (difference between an online and offline approach)?
	\item What are the particularities of our implementation (both model and datasets), what are its upsides and downsides (for example: requires huge datasets but fast inference)?
	\item What did we ultimately achieve? How good are our results?
\end{itemize}
\color{black}

\section{Introduction}

\color{red}
Concise presentation of the problem

*Note that layers will be explained in an upcoming section*

Preprocessing of text into phonemes?

SOTA ON MULTISPEAKER TTS:

First SPSS methods [2 - 20] of https://arxiv.org/pdf/1606.06061.pdf

\color{black}
Previous state of the art in TTS include hidden Markov models (HMM) based speech synthesis, which is a statistical parametric speech synthesis (SPSS) method. HMMs are trained to synthesize mel-frequency cepstral coefficients (MFCC) with energy, their delta and delta-delta coefficients \cite{TTSSOTA}. The result is passed through a vocoder\footnote{Specifically in TTS, some authors define a vocoder as a voice encoder that retrieves speech parameters to be used in synthesis. The more common definition however, is that of a function that generates a raw audio waveform from temporal features such as MFFC. This is the one we will use. \color{red} Review this \color{black}} such as MLSA \cite{MLSA}. The spectral parameters, pitch parameters and state durations of the model are conditioned on the phoneme contexts such that different contexts are clustered by a decision tree and a distribution is learned for each cluster \cite{HMMTTS}. It is thus possible to modify the voice generated by conditioning on a speaker or tuning these parameters with adaptation or interpolation techniques (e.g. \cite{HMMSpeakerInterpolation}), effectively making HMM-based speech synthesis a multispeaker TTS system. \color{red} Compare with concatenative \cite{SPSSDNN} ? \color{black}

\cite{SPSSDNN} proposed to model

\color{red}




Wavenet:

\color{black}
Breakthrough in TTS with raw waveform gen
https://deepmind.com/blog/wavenet-generative-model-raw-audio/ ??
Dilated causal convolutions
Condition on a speaker identity
\color{red}

Tacotron

Deep voice (1, 2, 3 + few samples), Tacotron 2

SV2TTS

Extensions?
\color{black}



\color{red}
\color{black}

\clearpage
\bibliographystyle{plain}
\bibliography{references} 




















%$$\Leftrightarrow h_b(x) =
%\left\{\begin{array}{lll}
%0 & if & P(y = 0 | x) > P(y = 1 | x)\\ 
%1 & else &
%\end{array}\right.$$



%\begin{figure}[h]
%	\centering
%	\includegraphics[width=16cm]{image.png}
%	\caption{caption}
%	\label{label}
%\end{figure}



%\begin{figure}[h]
%	\centering
%	\captionsetup{justification=centering}
%	\hspace{-1cm}
%	\subfigure{\includegraphics[height=5cm]{image.png}}
%	\subfigure{\includegraphics[height=5cm]{image.png}}
%	\hspace{-1cm}
%	\caption{caption}
%	\label{label}
%\end{figure}


%\begin{wrapfigure}{r}{5cm}
%	\vspace{-0.6cm}
%	\centering
%	\includegraphics[width=5cm]{image.png}
%	\caption{caption}
%	\label{label}
%	\vspace{-1.1cm}
%\end{wrapfigure}


%\begin{center}
%	\begin{tabular}{|r|ccc|ccc|}
%		\hline
%		& \multicolumn{6}{c|}{Validation set}\\
%		\hline
%		& \multicolumn{3}{c|}{Valid images (3126)} & \multicolumn{3}{c|}{Invalid images (3126)} \\
%		\hline
%		& Correct & Unclassified & Incorrect & Correct & Unclassified & Incorrect \\
%		\hline
%		Reduced & 94.98\% & 3.07\% & 1.95\% & 95.27\% & 2.91\% & 1.82\% \\
%		Lenet & 98.08\% & 0.74\% & 1.18\% & 97.86\% & 0.96\% & 1.18\% \\
%		\hline
%	\end{tabular}
%	
%	\vspace{0.5cm}
%	  
%	\begin{tabular}{|r|ccc|ccc|}
%		\hline
%		& \multicolumn{6}{c|}{Test set}\\
%		\hline
%		& \multicolumn{3}{c|}{Valid images (999)} & \multicolumn{3}{c|}{Invalid images (74)} \\
%		\hline
%		& Correct & Unclassified & Incorrect & Correct & Unclassified & Incorrect \\
%		\hline
%		Reduced& 94.29\% & 3.70\% & 2.00\% & 95.95\% & 4.05\% & 0.00\%  \\
%		Lenet & 96.90\% & 1.30\% & 1.80\% & 97.30\% & 1.35\% & 1.35\% \\
%		\hline
%	\end{tabular}
%\end{center}


\end{document}













































































































